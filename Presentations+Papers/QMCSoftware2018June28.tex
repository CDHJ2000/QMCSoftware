%Talk given at Matrix, June ??, 2018
\documentclass[11pt,compress,xcolor={usenames,dvipsnames},aspectratio=169]{beamer}
%\documentclass[xcolor={usenames,dvipsnames},aspectratio=169]{beamer} %slides and 
%notes
\usepackage{amsmath,datetime,
	mathtools,
	bbm,
	mathabx,
	array,
	booktabs,
	alltt,
	xspace,
	calc,
	colortbl,
 	graphicx}
   
\providecommand{\HickernellFJ}{H.}
\usepackage[autolinebreaks,useliterate]{mcode}
\usepackage[usenames]{xcolor}
\usepackage[backend=biber,style=nature]{biblatex}
\addbibresource{FJHown23.bib}
\addbibresource{FJH23.bib}
\usepackage{newpxtext}
\usepackage[euler-digits,euler-hat-accent]{eulervm}
\usepackage{media9}
\usetheme{FJHSlimNoFoot169}
\setlength{\parskip}{2ex}
\setlength{\arraycolsep}{0.5ex}

\renewcommand{\OffTitleLength}{-10ex}
\setlength{\FJHThankYouMessageOffset}{-8ex}
\title{A Proposal for Community (Quasi-) Monte Carlo Software}
\author[]{Fred J. Hickernell}
\institute{Department of Applied Mathematics \\
	Center for Interdisciplinary Scientific Computation \\  Illinois Institute of Technology \\
	\href{mailto:hickernell@iit.edu}{\url{hickernell@iit.edu}} \quad
	\href{http://mypages.iit.edu/~hickernell}{\url{mypages.iit.edu/~hickernell}}}
\thanksnote{Thanks to the workshop organizers, the GAIL team \\ NSF-DMS-1522687 and NSF-DMS-1638521 (SAMSI)
}
\event{Follow-Up to Discussions}
\date[]{June 28, 2018}

\input FJHDef.tex


\begin{document}
\everymath{\displaystyle}
\frame{\titlepage}

\section{Motivation}

\begin{frame}
\frametitle{Questions that I Ask or Hear Asked}

\vspace{-8ex}

\begin{itemize}[<+->]
\item Where can I find \alert{quality}, \alert{free} quasi-Monte Carlo (qMC) software?

\item Where can I find \alert{use cases} that illustrate how to employ qMC methods? 

\item How can I try that qMC \alert{method} developed by X's group for my problem?

\item How can I try my qMC method on the \alert{example} shown by Y's group?

\item How can my student get results without writing code from \alert{scratch}?

\item How can the code that I use benefit from \alert{recent} developments?

\item How can my work receive wider \alert{recognition}?
\end{itemize}

\only<+->{My initial attempts in this direction over the past 5 years have produced GAIL\footfullcite{ChoEtal17b}}

\end{frame}

\begin{frame}
\frametitle{Can We Have qMC Community Software that Grows Up to Be Like \ldots }

\vspace{-4ex}

\begin{description}
\item[Chebfun] Computing with Chebyhsev polynomials, \href{http://www.chebfun.org}{\nolinkurl{chebfun.org}}

\item[Clawpack] Solution of conservation laws, \href{http://www.clawpack.org}{\nolinkurl{clawpack.org}}

\item[deal.II] Finite-elements, \href{http://www.dealii.org}{\nolinkurl{http://dealii.org}} \\

{\small Mission: To provide well-documented tools to build finite element codes for a broad variety of PDEs, from laptops to supercomputers.

\alert{Vision: To create an open, inclusive, participatory community providing users and developers with a state-of-the-art, comprehensive software library that constitutes the go-to solution for all finite element problems.}}

\item[FEniCS] Finite-elements, \href{https://fenicsproject.org}{\nolinkurl{fenicsproject.org}}

\item[Gromacs] Molecular dynamics, \href{http://www.gromacs.org}{\nolinkurl{gromacs.org}}

\item[Stan] Markov Chain Monte Carlo, \href{http://mc-stan.org}{\nolinkurl{mc-stan.org}}

\item[Trilinos] Multiphysics computations, \href{https://trilinos.org}{\nolinkurl{trilinos.org}}

\end{description}

\vspace{-4ex}

\begin{itemize}
\item Developed and supported by multiple research groups

\item Used beyond the research groups that develop it

\item A recognized standard in its field 
\end{itemize}
\end{frame}

\section{Landscape}

\begin{frame}
\frametitle{What Is Available Now}

\vspace{-4ex}
{\small
\begin{description}
	\renewcommand{\itemsep}{-0.2ex}
	\renewcommand{\parsep}{0ex}

\item[John Burkhardt] Variety of qMC Software in C++, Fortran, MATLAB, and Python, \href{http://people.sc.fsu.edu/~jburkardt/}{\nolinkurl{people.sc.fsu.edu/~jburkardt/}}


\item[Mike Giles] Multi-Level Monte Carlo Software in C++, MATLAB, Python, and R, \href{https://people.maths.ox.ac.uk/gilesm/mlmc/}{\nolinkurl{people.maths.ox.ac.uk/gilesm/mlmc/}}

\item[Fred Hickernell] Guaranteed Automatic Integration Library (GAIL) in MATLAB,
\href{http://gailgithub.github.io/GAIL_Dev/}{\nolinkurl{gailgithub.github.io/GAIL_Dev/}}


\item[Stephen Joe \& Frances Kuo] Sobol' generators in C++, Generating vectors for lattices, 
\href{http://web.maths.unsw.edu.au/~fkuo/}{\nolinkurl{web.maths.unsw.edu.au/~fkuo/}}

\item[Pierre L'Ecuyer] Random number generators, Stochastic Simulation, Lattice Builder in C/C++ and Java,
\href{http://simul.iro.umontreal.ca}{\nolinkurl{simul.iro.umontreal.ca}}

\item[Dirk Nuyens] Magic Point Shop, QMC4PDE, etc. in MATLAB, Python, and C++,
\href{https://people.cs.kuleuven.be/~dirk.nuyens/}{\nolinkurl{people.cs.kuleuven.be/~dirk.nuyens/}}


\item[Art Owen] Various code,
\href{http://statweb.stanford.edu/~owen/code/}{\nolinkurl{statweb.stanford.edu/~owen/code/}}

\item[Christoph Schwab] Partial differential equations with random coefficients

\item[MATLAB] Sobol' and Halton sequences

\item[Python] Sobol' and Halton sequences

\item[R] \texttt{randtoolbox} Sobol', lattice, and Halton sequences

\end{description}}
\end{frame}

\section{Big Decisions}

\begin{frame}
\frametitle{Decisions to Make If We Want to Succeed}

\vspace{-7ex}

\begin{description}
\only<.-9>{\item[Key Elements] \
\begin{itemize}[<+-| alert@+>]
\item Sequences---IID, Sobol', lattice, Halton, sparse grid,  \ldots, including randomization; fixed and extensible sample size and dimension; constructions using optimization

\item Sequence generators---with inputs $d$, coordinate indices, and index range

\item Discrepancy measures---various kernels and domains

\item Variable transformations to accommodate non-uniform distributions and  domains other than the unit cube

\item Integrands, but some will come from external library, e.g., PDE solvers

\item Integrators, including multilevel and multivariate decomposition methods

\item Stopping criteria

\item Compelling use cases

\item Packages that display output in tables or plots
\end{itemize}}

\only<+-13>{\item[Languages and Architectures] \

\addtocounter{beamerpauses}{8}
\begin{itemize}[<+-| alert@+>]

\item Which one(s)?  Python?  C++?

\item How do we balance performance, developer time, and portability?

\item How will users connect the software with other software packages and environments?

\item How will parallel computing be supported?

\end{itemize}}

\only<+->{\item[Good Development Practices] \

\addtocounter{beamerpauses}{11}
\begin{itemize}[<+-| alert@+>]

\item Start small, with good skeleton

\item Version control on Git or equivalent

\item Ownership of routines, updates require owners' approval

\item Comprehensive tests run regularly

\item Reasonable license

\item Marketing on websites and at conferences

\end{itemize}}

\end{description}

\end{frame}

\section{Next Steps}

\begin{frame}
\frametitle{Bigger than One Research Group}

\vspace{-7ex}

\begin{itemize}
\item We have experience over the past 5 years developing GAIL\footfullcite{ChoEtal17b}

\item Our group has learned some of the discipline required to develop good software

\item Our group does not have the capacity to tackle this whole project, and neither does your group

\item A good software library should attract developers

\item Let's leave a legacy to our community that goes beyond theorems and algorithms
\end{itemize}


\end{frame}

\begin{frame}
\frametitle{We Must Weigh \ldots}

\vspace{-7ex}

\setlength{\extrarowheight}{1ex}
\begin{tabular}
{p{0.47\textwidth}p{0.47\textwidth}}
\alert{\Large Costs} & \alert{\Large Benefits} \tabularnewline
\toprule
Less time to prove theorems &
More impact for our theorems \tabularnewline
Learning a new language & 
Wider access and better performance for our code \tabularnewline
Compromise with other research groups &
More capable code than can be produced by one research group \tabularnewline
Time spent writing documentation and tests &
Fewer bugs for those who use the code \tabularnewline
&
Attract more qMC developers \tabularnewline
&
Attract more qMC users \tabularnewline
&
Happier funding agencies \tabularnewline

\end{tabular}
\end{frame}

\begin{frame}
\frametitle{Current Discussions}
\begin{itemize}[<+-| alert@+>]
\item If you are interested, whether you are a potential developer or user, let's talk

\item Determine what we can agree on as initial answers to the big questions

\item Right now we are trying to focus on 
\begin{itemize}
\item An initial language

\item An initial version control platform, and a model for collaboration
\item A few initial routines
\item An initial use case
\end{itemize}

\item We welcome you to join us.  Email me to join our Google group

\end{itemize}

\end{frame}

\section{Elements}

\begin{frame}[fragile]
\frametitle{Elements of the Community Software---Discrete Distributions}
\vspace{-5ex}
\lstinputlisting[lastline = 13]{discreteDistribution.m}
\end{frame}

\begin{frame}[fragile]
\frametitle{Elements of the Community Software---Discrete Distributions}
\vspace{-5ex}
\lstinputlisting[linerange = {1-1,14-21}]{discreteDistribution.m}
\end{frame}

\begin{frame}[fragile]
\frametitle{Elements of the Community Software---Discrete IID Uniform}
\vspace{-5ex}
\lstinputlisting[lastline = 14]{IIDDistribution.m}
\end{frame}

\begin{frame}[fragile]
\frametitle{Elements of the Community Software---Discrete IID Uniform}
\vspace{-5ex}
\lstinputlisting[linerange = {1-1,16-30}]{IIDDistribution.m}
\end{frame}

\begin{frame}[fragile]
\frametitle{Elements of the Community Software---Functions}
\vspace{-5ex}
\lstinputlisting{fun.m}
\end{frame}

\begin{frame}[fragile]
\frametitle{Elements of the Community Software---Keister's Function}
\vspace{-5ex}
\lstinputlisting{KeisterFun.m}
\end{frame}

\begin{frame}[fragile]
\frametitle{Elements of the Community Software---StoppingCriteria}
\vspace{-5ex}
\lstinputlisting{stoppingCriterion.m}
\end{frame}

\begin{frame}[fragile]
\frametitle{Elements of the Community Software---CLTStopping}
\vspace{-5ex}
\lstinputlisting[lastline=12]{CLTStopping.m}
\end{frame}

\begin{frame}[fragile]
\frametitle{Elements of the Community Software---CLTStopping}
\vspace{-5ex}
\lstinputlisting[linerange={1-1,14-20}]{CLTStopping.m}
\end{frame}


\begin{frame}[fragile]
\frametitle{Elements of the Community Software---CLTStopping}
\vspace{-5ex}
\lstinputlisting[linerange={1-1,20-34}]{CLTStopping.m}
\end{frame}


\begin{frame}[fragile]
\frametitle{Elements of the Community Software---CLTStopping}
\vspace{-5ex}
\lstinputlisting[linerange={1-1,35-44}]{CLTStopping.m}
\end{frame}

\begin{frame}[fragile]
\frametitle{Elements of the Community Software---CLTStopping}
\vspace{-5ex}
\lstinputlisting[linerange={1-1,45-61}]{CLTStopping.m}
\end{frame}

\begin{frame}[fragile]
\frametitle{Elements of the Community Software---Integration}
\vspace{-5ex}
\lstinputlisting{integrate.m}
\end{frame}

\begin{frame}[fragile]
\frametitle{Elements of the Community Software---Integration Example}
\vspace{-5ex}
\lstinputlisting[lastline=6]{IntegrationExample.m}

\begin{alltt}
>> IntegrationExample
sol = 0.4310
out = 
timeUsed: 0.0014
nSamplesUsed: 7540
errorBound: [0.4210 0.4410]
\end{alltt}
\end{frame}

\begin{frame}[fragile]
\frametitle{Elements of the Community Software---Integration Example}
\vspace{-5ex}
\lstinputlisting[linerange={7-8}]{IntegrationExample.m}

\begin{alltt}
sol = 0.4253
out = 
timeUsed: 0.0333
nSamplesUsed: 652546
errorBound: [0.4243 0.4263]
\end{alltt}
\end{frame}

\begin{frame}[fragile]
\frametitle{Elements of the Community Software---Integration Example}
\vspace{-5ex}
\lstinputlisting[linerange={9-11}]{IntegrationExample.m}

\begin{alltt}
sol = 0.4252
out = 
timeUsed: 0.0392
nSamplesUsed: 1000000
errorBound: [0.4244 0.4260]
\end{alltt}
\end{frame}


\begin{frame}[fragile]
\frametitle{Elements of the Community Software---Asian Call, Low D}
\vspace{-5ex}
\lstinputlisting[linerange={13-18}]{IntegrationExample.m}

\begin{alltt}
OptionObj = 
dimension: 4
sol = 6.1740
out = 
timeUsed: 1.0517
nSamplesUsed: 5680546
errorBound: [6.1640 6.1840]
\end{alltt}
\end{frame}


\begin{frame}[fragile]
\frametitle{Elements of the Community Software---Asian Call, High D}
\vspace{-5ex}
\lstinputlisting[linerange={20-21}]{IntegrationExample.m}

\begin{alltt}
OptionObj = 
AsianCallFun with properties:
dimension: 64
sol =6.2036
out = 
meanVarData with properties:
timeUsed: 25.1910
nSamplesUsed: 5610402
errorBound: [6.1936 6.2136]
\end{alltt}
\end{frame}


\begin{frame}[fragile]
\frametitle{Elements of the Community Software---Asian Call, Multi-Level}
\vspace{-5ex}
\lstinputlisting[linerange={23-24}]{IntegrationExample.m}

\begin{alltt}
OptionObj = 
1×3 AsianCallFun array with properties:
sol = 6.2052
out = 
timeUsed: 2.2171
nSamplesUsed: [8080862 446720 85907]
errorBound: [6.1968 6.2135]
\end{alltt}
\end{frame}








\finalthanksnote{Slides available on SpeakerDeck at
	\href{https://speakerdeck.com/fjhickernell/qmc-software-presentation-to-gail}
	{\nolinkurl{speakerdeck.com/fjhickernell/qmc-software-presentation-to-gail}} 
    
}

\thankyouframe

\printbibliography



\end{document}